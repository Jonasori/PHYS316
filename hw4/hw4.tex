\documentclass[12pt]{article}
\usepackage[pdftex]{graphicx}
%\usepackage[dvips]{graphicx}
\usepackage{latexsym}
\usepackage{multicol}
\usepackage{physics}
\usepackage{amsfonts}
\usepackage{amssymb}
%\usepackage{braket}

%margin control
%\renewcommand{\baselinestretch}{1.1}
\usepackage{geometry}
\geometry{textheight=9in,
          textwidth=6.5in,
          top=1.0in,
          left=1.0in}


\usepackage[compact,sf,bf]{titlesec}
\usepackage[lflt]{floatflt}

\setlength{\headheight}{30pt}
\usepackage{wrapfig}
\usepackage{fancyhdr}
\pagestyle{fancyplain}

\usepackage[usenames,dvipsnames]{xcolor}
\usepackage{comment}

\newenvironment{answer}[2][Solution]{\begin{trivlist}
\item[\hskip \labelsep {\bfseries #1}\hskip \labelsep {\bfseries #2.}]}{\end{trivlist}}


\lhead{{\sf \textbf{{\large PHYS 316: Themal \& Statistical Physics}\\
      Jonas Powell} }} \rhead{
\begin{floatingfigure}[r]{1cm}
\vspace{-0.2cm}\includegraphics[height=1cm]{/Users/jonas/Downloads/wes_shield_lores.png}
\end{floatingfigure}
{\sf \textbf{Homework 4} \\
  \textbf{Due: Thursday February 22}}}

\begin{document}

\begin{description}


\item[Problems:]
\end{description}

\begin{enumerate}

%\item Baierlein 5.2 -- three state system

%\item Schroeder 6.13
\item Last week we calculated the high and low $T$ behavior of the
  quantum oscillator from a microcanonical (fixed $NVE$) perspective,
  where we count states with specific energy.  This week, we
  tackle the same question, but from the perspective of the canonical
  partition function (fixed $NVT$), where we take advantage of the
  partition function machinery.

Recall that the energy of a quantum oscillator is given by
$$E_n = \left(n+\frac{1}{2}\right) \epsilon_0 $$
where $\epsilon_0/2$ is the ground state energy, and $n$ is an integer
from zero to infinity describing the occupied energy level.


% 1A
\begin{enumerate}
\item Calculate the partition function $Z$ for the quantum oscillator.
  You will find it helpful to remember the expression for an infinite
  geometric sum.

  \begin{answer}{1a}


    We begin with three useful equations: first, the equations dictating the energetic steps of a quantum oscillator; second, the infinite geometric sum identity; and third, the partition function definition.

    \begin{equation}
      E_{oscillator} = \epsilon_0 (n \ + \ \frac{1}{2})
    \end{equation}

    \begin{equation}
      \sum_{i=0}^{\inf} a \cdot r^i = \frac{a}{1-r}
    \end{equation}

    \begin{equation}
      Z = \sum_{states} e^{- \beta E_{state}} = \sum_{states} e^{- \frac{E_{state}}{kT}}
    \end{equation}

    Substituting (1) into (3), we find:
    \begin{equation}
      Z = \sum_{states} e^{- \beta E_{state}} = \sum_{states} e^{- \frac{E_{state}}{kT}}
    \end{equation}

  \end{answer}








\item From the partition function, evaluate the mean energy $U$.
\item Evaluate the $T\rightarrow 0$ and $T\rightarrow \infty$ limits
  of the energy.  Check that your results are consistent with the
  previous week.  Isn't this approach much nicer?
\end{enumerate}

\item Schroeder 6.22, parts (a) and (b) only.  Be sure that you understand
  the simple paramagnet described on p.\ 232-3.



\item In the ``Zipper'' model for the formation of double-standed DNA
  from two single strands, the cooperativity of bonding requires that
  strands zip (or unzip) only from one end, so that the $n^{th}$ link
  between strands only forms if bonds $(1, 2, \dots, n-1)$ are all
  intact, mimicking a zipper.  For each intact bond, $E = -\epsilon$,
  while each open bond has $E=0$.
\begin{enumerate}
\item For DNA of length $N$ bases, show that the partition function is given by
$$ {Z} = \frac{e^{(N+1)\beta\epsilon}-1}{e^{\beta\epsilon}-1}. $$
The finite geometric series you proved in Schroeder 6.22 is helpful.
\item Determine the average number of bonded links $\langle n \rangle$.
  To calculate this, take advantage of the fact that the average number
  of links $\langle n \rangle = -U/\epsilon$.
\item For $T=0$, all links must be intact.  For $kT \ll \epsilon$ ({\it
    i.e.}\ low $T$, but not zero), determine the average number of bonded
  links to first order in $e^{-\beta\epsilon}$.  To do so, you will want
  to take advantage of that $e^{-\beta\epsilon} \ll 1$ at low
  temperature.
\end{enumerate}

\item Schroeder 6.26.  Including rotations is a step toward extending
  the ideal gas description to simple molecular systems, like H$_2$ or
  N$_2$.  Be sure you understand Schroeder's discussion from p.\ 234-236.

\begin{comment}
%% Schroeder does high T limit
%% Low T limit done in question 6.26
\item To extend our treatment of the ideal gas to molecular systems, we
  need to account for the fact that molecules can rotate.  From quantum
  mechanics, we know that rotational energy levels are quantized
  according to
\[ \epsilon(j) = j(j+1)\epsilon \]
where $j=0,1,2,...$.  In addition, each rotational level has a multiplicity
\[g(j) = 2j+1. \]
\begin{enumerate}
\item Evaluate the partition function for the rotational motion.
  Remember that you must sum over all {\it states}, and that each energy
  level has multiplicity.
\item In the high-$T$ limit ($kT \gg \epsilon_{0}$), evaluate $Z$ by
  converting the sum to an integral.
\item In the low-$T$ limit ($kT \ll \epsilon_{0}$), evaluate $Z$ by
  truncating the sum after the first non-trivial term (one is trivial).
\item Evaluate the energy $U$ in both the high- and low-$T$ limits.
\item Sketch $U$ as a function of $T$.  Note that I really mean sketch,
  do not use a computer.  You will have only a low- and high-$T$
  expression, so you will have to connect the two in a reasonable way.
\end{enumerate}
\end{comment}

 \item Schroeder 6.51.
% \item Schroeder 6.52


\begin{comment}
%%NEXT WEEK
\item This problem illustrates an important point that the thermodynamic
  repsonse functions (like specific heat and compressibility) are
  related to the fluctuations around the mean of the thermodynamic
  variables.  We will illustrate this specifically for specific hear.

  Consider a system with fixed $NVT$ (canonical ensemble).  We already
  know that the mean energy
$$ U = \langle E \rangle = \frac{\sum_{s} E_s e^{-\beta E_s}}{Z}. $$
Show that the
  specific heat is related to fluctuations of energy via
$$ C_{V} = \frac{\partial \langle E \rangle}{\partial T} = \frac{1}{kT^{2}} \left( \langle E^{2} \rangle - \langle E \rangle^{2} \right). $$
The fact that specific heat (or any other second derivative of the free
energy) can be related to fluctuations is very valuable to computer
simulations of model systems.
\end{comment}



\end{enumerate}

\end{document}
